\documentclass[a4paper,twocolumn,openany,nodeprecatedcode]{dndbook}

% Use babel or polyglossia to automatically redefine macros for terms
% Armor Class, Level, etc...
% Default output is in English; captions are located in lib/dndstrings.sty.
% If no captions exist for a language, English will be used.
%1. To load a language with babel:
%	\usepackage[<lang>]{babel}
%2. To load a language with polyglossia:
%	\usepackage{polyglossia}
%	\setdefaultlanguage{<lang>}
\usepackage[english]{babel}
%\usepackage[italian]{babel}
% For further options (multilanguage documents, hypenations, language environments...)
% please refer to babel/polyglossia's documentation.

\usepackage[utf8]{inputenc}
\usepackage[singlelinecheck=false]{caption}
\usepackage{lipsum}
\usepackage{listings}
\usepackage{shortvrb}
\usepackage{stfloats}

\captionsetup[table]{labelformat=empty,font={sf,sc,bf,},skip=0pt}

\MakeShortVerb{|}

\lstset{
  basicstyle=\ttfamily,
  language=[LaTeX]{TeX},
  breaklines=true,
}

\title{Harry Potter 5e}
\author{Jens Keim}
\date{\today}

\begin{document}

\frontmatter

\maketitle

\tableofcontents

\mainmatter

\small

\chapter{Heritage}

    A character always gains the following benefits, including Ability Score Increases and Languages.

    \subparagraph{Size} Unless otherwise stated, your size is Medium.

    \subparagraph{Speed} Unless otherwise stated, your speed is 30 feet.

    \subparagraph{Ability Scores}
        You gain 3 points you can distribute among your ability scores. You can only add up to 2 points to any ability score.
        % When determining your character's ability scores, increase one score by 2 and increase a different score by 1, or increase three different scores by 1.

    \subparagraph{Languages}
        Your character can speak, read, and write English and one other language that you and your DM agree is appropriate for the character.

        % FIXME: add language table

        \begin{DndTable}[width=\textwidth, header=Standard Languages]{lll}
            \textbf{Language} & \textbf{Typical Speakers} & \textbf{Script} \\
            Bulgarian   & Bulgarians            & Cyrillic \\
            Cantonese   & Chinese (north, west) & Hanzi \\
            English     & British, Americans    & Latin \\
            French      & French                & Latin \\
            German      & Germans               & Latin \\
            Hebrew      & Jews                  & Hebrew \\
            Japanese    & Japanese              & Kanji, Kana \\
            Latin       & —                     & Latin \\
            Mandarin    & Chinese (southeast)   & Hanzi \\
            Spanish     & Spanish               & Latin
        \end{DndTable}

        \begin{DndTable}[width=\textwidth, header=Exotic Languages]{lll}
            \textbf{Language} & \textbf{Typical Speakers} & \textbf{Script} \\
            Gobbledegook      & Goblins         & ? \\
            Mermish           & Merpeople       & ? \\
            Parseltongue      & Wizards         & — \\
            Old English       & —               & Runes \\
            Troll             & Trolls          & — \\
        \end{DndTable}

        % Gaelic
        % Saxon


    \section{Metamorphmagus}
        % source: Wands & Wizardry > Metamorph Magic

        Once in a great many years, a metamorphmagus is born to a wizarding family with their very particular talent: morphing every aspect of their human appearance. Before becoming an adult, a metamorphmagus will not have complete control over this ability, often letting their emotions or stress get the better of them and losing control.

        \subparagraph{Shapechanger} At will, you can transform your appearance. As an action, you decide what you look like, including your height, weight, facial features, sound of your voice, hair length, coloration, and distinguishing characteristics, if any. None of your statistics change, you don't appear as a creature of a different size than you, and your basic shape stays the same. If you're bipedal, you can't use this spell to become quadrupedal, for instance. At any time, you can use your action to change your appearance in this way again.

        \subparagraph{Aquatic Adoption} You can also adapt your body to an aquatic environment, growing webbing between your fingers. As an action, you gain a swimming speed equal to your walking speed.


    \section{Muggles \& Squibs}

        \subparagraph{Adapted} When determining your character's ability scores, increase one score by 2 and increase a different score by 1, or increase three different scores by 1. (This is in addition to the Ability Score Increase your character gets regardless of their heritage.)

        \subparagraph{Mundane} You can not gain levels in any spell casting class. And are therefore restricted to martial classes.

        \subsection{Muggle}

            Muggles are not able to produce magic of any kind.
            Most live their entire lives without encountering any sign of magic, thanks to the International Statute of Wizarding Secrecy since 1682.

            However, it was possible that Muggles or No-Majes became aware of the wizarding world.
            Wether a relative turned out to be a Muggle-Born Wizard, you stumbled to deep into a magical forest, witnessed a magical event and escaped the obliviation, or you are a government official, who is in the know, now you are in it for the long haul.
            Supporting magically gifted or investigating this strange world for yourself.

            \subparagraph{Muggle Knowledge} You gain proficiency in \textit{Muggle Studies}.

        \subsection{Squib}

            A squib was a person born into a wizarding family, but was unable to perform magic themselves.
            Compared to muggles, squibs were aware of Wizarding World.
            Though most migrate to the muggle world and pursue non-magical education and carriers.
            Even though it was possible for them to study magic theoretically, magical schools would usually not admit them.

            \subparagraph{Knowledge} You gain proficiency in either \textit{History of Magic} or \textit{Muggle Studies}.

        \begin{DndSidebar}{Playing as a squib in Hogwarts}
            When playing in the larger wizarding world,
            playing a squib or even a muggle character is not a big problem.
            Since the muggle world and the wizarding world coexist and intertwine,
            it is not unusual to see a muggle finding their way in the midst of wizards by accident.

            However, when playing at a wizarding school, it is a bigger challenge to explain.
            Let me present some exceptional situations, that would allow a squib to attend Hogwarts or another wizarding school.
            \begin{itemize}
                \item They come from a wealthy and influential wizard family.
                      And they pay for the school to admit him anyway.
                \item They are invited as a guest speaker in \textit{Muggle Studies}.
                      (This could be temporary thing for a one-shot or a regular thing to include the character in the main events of a larger adventure.)
                \item They do some manual task around the castle, like Argus Filch.
                \item Maybe they were brought in to be part of a study on squibs or muggles.
                      (e.g. during the year of Snape's headmastership, in an earlier epoch, or more advanced epoch)
            \end{itemize}
        \end{DndSidebar}

    \section{Part-Elf}
        % original

        % source: Elf (High)
        \subparagraph{Cantrip} You know one cantrip of your choice from the wizard spell list. Intelligence is your spellcasting ability for it.

        \subparagraph{Wandless Magic} You gain the Wandless Magic Feat.


    \section{Part-Giant}
        % source: Wands & Wizardry > Giant's Blood

        Even though they are few and far between, it's hard for half-giants and part-giants to hide their genealogy; they tend to turn heads wherever they go. With a broad build, impressive strength and a genetic resistance to magic, wizards with giant blood are powerful allies.

        % source: Goliath MPMM
        \subparagraph{Size} Adult part-giants are between 7 and 9 feet tall and weigh between 300 and 420 pounds. Your size is Medium.

        \subparagraph{Little Giant} You have proficiency in the Athletics skill, and you count as one size larger when determining your carrying weight and the weight you can push, drag, or lift.

        \subparagraph{Mountain Born} You have resistance to cold damage. You also naturally acclimate to high altitudes, even if you've never been to one. This includes elevations above 20,000 feet.

        % \subparagraph{Stone's Endurance} You can supernaturally draw on unyielding stone to shrug off harm. When you take damage, you can use your reaction to roll a d12. Add your Constitution modifier to the number rolled and reduce the damage by that total.\\
        % You can use this trait a number of times equal to your proficiency bonus, and you regain all expended uses when you finish a long rest.

        % source: Wands & Wizardry > Giant's Blood
        \subparagraph{Spell Resistance} When a spell or other magical effect inflicts a condition on you, you can use your reaction to resist one condition of your choice. You can't use this ability again until you finish a long rest.


    \section{Part-Goblin}
        % source: Wands & Wizardry > Goblin Cunning

        The rarest racial combination of all, part-goblins can have goblin ancestry anywhere in their family tree, and it will still make a very noticeable difference. At an average of 4 feet tall, part-goblins may feel out of place in a larger world. However, these small wizards have inherited goblins' cleverness and often come with big hearts and big personalities.

        \subparagraph{Size} Adult part-goblins are between 3 and 5 feet tall and weigh around 110 pounds. Your size is Small.

        % source: Gnome
        \subparagraph{Goblin Cunning} You have advantage on all Intelligence, Wisdom, and Charisma saving throws against magic.

        % FIXME: instead give darkvision?
        % source: Halfling
        \subparagraph{Goblin Nimbleness} You can move through the space of any creature that is of a size larger than yours.


    \section{Part-Vampire}
        % source: VRGR > Dhampir

        \subparagraph{Speed} 35 ft., climb equal to your walking speed

        \subparagraph{Size} You are Medium (or Small). You choose the size when you gain this lineage.

        \subparagraph{Ancestral Legacy} You gain proficiency in two skills of your choice.

        \subparagraph{Darkvision} You can see in dim light within 60 feet of you as if it were bright light and in darkness as if it were dim light. You discern colors in that darkness as shades of gray.

        \subparagraph{Deathless Nature} You don't need to breathe.

        \subparagraph{Spider Climb} You have a climbing speed equal to your walking speed. In addition, at 3rd level, you can move up, down, and across vertical surfaces and upside down along ceilings, while leaving your hands free.

        \subparagraph{Vampiric Bite} Your fanged bite is a natural weapon, which counts as a simple melee weapon with which you are proficient. You add your Constitution modifier, instead of your Strength modifier, to the attack and damage rolls when you attack with this bite. It deals 1d4 piercing damage on a hit. While you are missing half or more of your hit points, you have advantage on attack rolls you make with this bite.\\
        When you attack with this bite and hit a creature that isn't a Construct or an Undead, you can empower yourself in one of the following ways of your choice:

        \begin{itemize}
            \item You regain hit points equal to the piercing damage dealt by the bite.
            \item You gain a bonus to the next ability check or attack roll you make; the bonus equals the piercing damage dealt by the bite.
        \end{itemize}

        You can empower yourself with this bite a number of times equal to your proficiency bonus, and you regain all expended uses when you finish a long rest.


    \section{Part-Veela}
        % source: Wands & Wizardry > Veela Charm

        Just like part-giants, it's very rare to find a part-veela. A wizard or witch who inherited veela blood will almost always be the center of attention, a picture of grace and beauty. Unlike half-veela or quarter-veela, part-veela are just as likely to be male as female.

        \subparagraph{Veela Instincts} You gain proficiency in one Charisma skill of your choice.
        \subparagraph{Veela Charm} As an action, you can attempt to charm a humanoid you can see within 30 ft, who would be attracted to you. It must make a Wisdom saving throw, (if hostile, with advantage). If it fails, it is charmed by you for one hour or until you or your companions harm it. The charmed creature regards you as a friendly acquaintance and feels compelled to impress you or receive your attention. After, the creature knows it was charmed by you. You can't use this ability again until you finish a long rest.


    \section{Wizard}
    % source: human (variant)

        \subparagraph{Skills} You gain proficiency in one skill of your choice.
        \subparagraph{Feat} You gain one feat of your choice.


        \subsection{Muggle-Born}

        \subparagraph{Muggle Knowledge} You gain proficiency in \textit{Muggle Studies}.


        \subsection{Half-Blood}


        \subsection{Pureblood}


\chapter{Feats}

    \subsection{Aerial Combatant}
    % source: W&W > Aerial Combatant

        You're able to keep yourself oriented and lead your targets while flying a broomstick.
        You gain the following benefits.

        \begin{itemize}
            \item You gain tool proficiency in Vehicles (Broomstick).
            % at least two more skill or tool proficiencies or another equivalent bonus
            % \item You no longer suffer disadvantage on attack rolls due to flying.
            % you should not suffer from disadvantage in the first place
        \end{itemize}


    \subsection{Legilimens}
    % source: OC

        You were born a legilimens and possess the ability to read emotions and thoughts.\\

        You learn the \textit{lesser legilimens} spell.
        It does not count against your number of known spells.
        Using this feat, you can cast the spell once at its lowest level, and you must finish a long rest before you can cast it in this way again.\\

        Additionally, you can read surface thoughts of your surroundings as described in the \textit{lesser legilimens} without casting the spell.
        You can do this a number of times equal to your proficiency bonus, and you regain all expended uses when you finish a long rest.


    \subsection{Parseltongue}
    % source: Wands & Wizardry > Parseltongue

        % If you are a at least a half-blood wizard you can be born a Parseltongue and gain the additional language.
        \textit{Prerequisite: Half-Blood or Pureblood Wizard Heritage}

        Almost exclusively hereditary, to speak Parseltongue is to magically comprehend and verbally communicate with all snakes and snake-like beasts, like the Runespoor and Basilisk. This oral language has been associated with Dark wizards, owing to Salazar Slytherin's status as a Parselmouth which was passed on to the Gaunt family and Tom Riddle. However, outside of Wizarding Britain, no such association exists.

        \subparagraph{Language} You can speak Parseltongue.

        \subparagraph{Slithering friends}
        No matter where you find yourself,
        snakes and snake-like beasts tend to find you.
        They enjoy your company.
        As long as, you and your allies do not harm them, they will provide you aid.


    \subsection{Raised by Wizards}
    % source: PHB > Magic Initiate

        You learn two cantrips of the sorcerer's spell list.\\
        In addition, choose one 1st-level spell to learn from that same list.
        Using this feat, you can cast the spell once at its lowest level, and you must finish a long rest before you can cast it in this way again.\\

        Your spellcasting ability for these spells depends on your casting type.


    \subsection{Wandless Magic}
    % source: Wands & Wizardry > Wandless Magic

        Through studying ancient tomes or channeling some of the volatile magic of your youth, you're able to perform small magical feats without your wand.
        If you know any of the following spells, you can cast them without needing a wand or somatic component: \textit{accio, alohomora, colovaria, illegibilus, incendio glacia, pereo, wingardium leviosa}.\\
        You cannot expend a higher level spell slot when casting in this way.


\end{document}
